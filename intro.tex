Hello. This is going to be a good paper.

Access to clean water is considered a basic human right. While almost all developed countries have made clean water readily accessible to all their citizens, it is estimated that 663 million people still living without consistent access to sanitary water (cite the UN).
This issue has been a major focus for the United Nations, which set a goal in of making clean water accessible to every individual by the year 2030.
In recent years, significant work and progress has been made in improving water sanitation and availability to water poor countries, but now the most obvious improvements have been made.
With this in mind, there are still many regions and communities that need access to this basic right.
And our research focused on buidling a model to use the available data to help find the regions most in need of a water intervention. 
[Emphasize how hard it is to reach remaining population] 

We had a a research project focused on developing a model to help find the best ways to improve the water sources and water quality of a country.
Our approach was to rank each regional location, which we will call clusters, based on the ratio of expected benefits from a water intervention per dollar spent.
This model is meant to aid countries and charitable organizations in their decision making process for selecting locations that may need the most help.
Our approach centers on taking the Disability Adjusted Life Years calculations, developed by the World Health Organization, and applying them to the survey data provided by the Demographics and Health Survey.
This was a large source of data that not being fully implemented(can we prove or quantify this).
Creating an easily accessible model that helps these organizations understand and apply this large source of valuable information was the central goal.  

As an overview, our water intervention model is focused on determining the best interevention for each of the clusters in a given country.
There are a handful of types of interventions including Chlorine, Dug well, Bore hole, and stand pipes.
The intervention with the best benefit to cost ratio for the given cluster was chosen, and this made possible a comparison of every cluster in the country.
Equipped with this model and information, charitable organizations are able to more rapidly and confidently select the best regions to work in next.

[Make sure these second two paragraphs info isn't repeated in later sections]
[emphasize the importance of making a model like this]
