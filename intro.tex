Access to clean water is considered a basic human right.
While almost all developed countries have made clean water readily accessible to all their citizens, it is estimated that 663 million people currently live without consistent access to sanitary water (cite the UN).
This issue has been a major focus for the United Nations, which set a goal to make clean water accessible to every individual by the year 2030 (insert reference).
As a result, significant work and progress has been made in recent years to improve water sanitation and availability in water poor countries, but now the most obvious improvements have been made.
With this in mind, there are still many regions and communities that lack access to this basic human right.
Resolving this issue is rather difficult, because it attempts to reach that small percentage of people that is usually overlooked in the initial push for intervention.
It is further complicated by the fact that these communities are often difficult to identify without personally visiting them first.
However, attempting to visit all such locations is impractical at best, if not infeasible.
As such, the focus of this research is on building a mathematical model to identify the regions most in need of a water intervention, based on currently available data. 

The model was created to be an aid to countries and charitable organizations in selecting suitably impactful locations for a water intervention.
It is designed to find the best ways to improve the water sources and water quality in a given country.
It accomplishes this by ranking each regional location, or cluster, based on the ratio of expected benefits in Disability Adjusted Life Years (DALYs) from a water intervention per dollar spent (DALY per dollar).
Included is a case study which takes DALY calculations, developed by the World Health Organization, and applies them to survey data provided by The Demographics and Health Survey program (the DHS program).
The model provides an easy way for governments and charitable organizations to more fully utilize this valuable source of information and to make notable strides in improving water accessibility and quality.

Considered as interventions are a handful of options including chlorine filter distribution, dug wells, bore holes, and stand pipes.
The interventions were then ranked by the benefit to cost ratio for the given cluster, the optimal intervention chosen, and then compared with the best intervention of every other cluster in the country.
Equipped with this model and information, charitable organizations will be able to more rapidly and confidently select the best regions to work in next.

With the described analysis framework in this paper, government agencies can be assured that their actions to improve water accessibility and quality are the most effective decisions possible.
It also facilitates reporting and encourages accurate data collection and compilation for nations and regions.
Ultimately, its purpose is to create an accurate and scalable model for the ranking of water interventions that can be used as a tool for decision making and reporting.

[Make sure info isn't repeated in later sections]
