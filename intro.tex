Access to clean water is considered a basic human right.
While almost all developed countries have made clean water readily accessible to all their citizens, it is estimated that 663 million people currently live without consistent access to sanitary water (cite the UN).
This issue has been a major focus for the United Nations, which set a goal to  make clean water accessible to every individual by the year 2030.
As a result, significant work and progress has been made in recent years to improve water sanitation and availability in water poor countries, but now the most obvious improvements have been made.
With this in mind, there are still many regions and communities that lack access to this basic human right.
Resolving this issue is rather difficult, because it attempts to reach that small percentage of people that is usually overlooked in the initial push for intervention.
It is further complicated by the fact that these communities are often difficult to identify without personally visiting them first.
However, attempting to visit all such locations is impractical at best, if not infeasible.
As such, the focus of this research is on building a mathematical model to identify the regions most in need of a water intervention, based on currently available data. 

The model was created to be an aid to countries and charitable organizations in selecting suitably impactful locations for a water intervention.
It is designed to find the best ways to improve the water sources and water quality in a given country.
It accomplishes this by ranking each regional location, or cluster, based on the ratio of expected benefits in Disability Adjusted Life Years (DALYs) from a water intervention per dollar spent (DALY per dollar).
It takes the DALY calculations, developed by the World Health Organization, and applys them to survey data provided by The Demographics and Health Survey Program (the DHS program).
The model provides an easy way for countries and charitable  organizations to more fully utilize this valuable source of information.

There are a handful of types of interventions including chlorine distribution, dug wells, bore holes, and stand pipes.
The intervention with the best benefit to cost ratio for the given cluster was chosen, and then a comparison was made between every cluster in the country, thus determining the rankings.
Equipped with this model and information, charitable organizations will be able to more rapidly and confidently select the best regions to work in next.

[Make sure these second two paragraphs info isn't repeated in later sections]
[emphasize the importance of making a model like this]
