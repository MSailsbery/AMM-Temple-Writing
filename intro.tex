Access to clean water is considered an essential human right, but an estimated 884 million people currently lack access to safe drinking water (see UN Resolution 64/292).
To ensure the availability of water for all, the 2015 United Nations General Assembly resolved to provide universal access to clean water by 2030 (see UN Resolution 70/1).
As a result of this resolution, and in conjunction with the Millennial Development Goals created in 2000, water-poor countries have made significant progress in recent years to improve water sanitation and availability.
Despite these improvements, there are still many regions and communities that lack access to this essential human right.
Resolving this issue of expanding clean water capabilities to those water-poor regions and communities is rather difficult because it attempts to reach that small percentage of people that is usually overlooked in the initial push for intervention due to location or population.
It is further complicated by the fact that these communities are often difficult to identify without personally visiting them first; however, attempting to visit all such locations is impractical, if not infeasible.
As such, the focus of this research is on building a mathematical model to identify the regions most in need of a water intervention, based on data that is currently available. 

The model was created to aid countries and charitable organizations in selecting suitable locations for impactful water interventions.
It is designed to find the best ways to improve the water sources and water quality in a given country.
% From Katie: So, I'm not super familiar with the model, but are the results a location or a method or both?
It accomplishes this by ranking each regional location, or cluster, based on the ratio of expected benefits in Disability Adjusted Life Years (DALYs) from a water intervention per dollar spent (DALY per dollar).
Included is a case study that takes DALY calculations developed by the World Health Organization, and applies them to survey data provided by the Demographics and Health Survey program (the DHS program).
The model provides an easy way for governments and charitable organizations to more fully utilize this valuable source of information and to make notable strides in improving water accessibility and quality.
% Katie note to self: find a way to reword this last sentence. Also, what are we considering the "valuable source of information" here?

Considered as interventions within the model are a handful of options including chlorine filter distribution, dug wells, bore holes, and stand pipes.
The interventions are ranked by the benefit to cost ratio for the given cluster, the optimal intervention chosen, and compared with the best intervention of every other cluster in the country.
% From Katie: Will someone a little more familiar with the project details review this sentence and make it a little more clear what is trying to be said?
Equipped with this model and information, charitable organizations can more rapidly and confidently select the best regions to initiate water improvement projects.

With the described analysis framework in this paper, government agencies can be assured that their actions to improve water accessibility and quality are the most effective decisions possible.
This paper also facilitates reporting and encourages accurate data collection and compilation for nations and regions.
% From Katie: How does this paper facilitate reporting?
Ultimately, its purpose is to create an accurate and scalable model for the ranking of water interventions that can be used as a tool for decision making and reporting.

[Make sure info isn't repeated in later sections]
