We have presented a technique for modeling water infrastructure that predicts the social benefit gained by making an intervention. We also have shown how this technique can be used to inform decisions by governmental bodies and NGO's alike in how to help the most people the fastest.

This model has assumptions that could be improved significantly with more data.
Specifically, the spatial data of respondents to the DHS program survey.
The more data that the humanitarian aid and research communities can compile, the stronger, and more efficient optimization techniques like this one will be in assessing where to enact water infrastructure interventions.

In its current form, it is the researchers' opinion that this model would be best implemented as a metaphorical sieve.
It can be used to identify the top several water infrastructure interventions in an area of the country, which then can be further analyzed by experts through field work in order to quickly find the most beneficial intervention.

The AMM research team is also working on a technique to identify the most efficient order in which to enact the water infrastructure changes mentioned here.
Solving this problem would give practitioners a tool to understand the amount of benefit a certain water infrastructure budget would give to citizens.
