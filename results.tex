[Line up with Journal of Water and Health]

The main set of results given by this method comes from a case study done on Namibia. This work was facilitated by the DHS survey data from 2012.

\subsection{Case Study: Namibia}

Although many countries in sub-saharan Africa have extreme need for better access to improved water sources******,  Namibia already has reached the point where improved water sources are available to 79\% of its citizens******(JMP 2017 final report).
Thus the goal of this case study was to identify the regions and communities that still remain in need of further improvements in their water sources.  

The first noteworthy result was that in almost every cluster the most cost-effective water intervention was the distribution chlorine filters. [Talk about this in the discussion, possibly moving some of this discussion].
This was the cheapest and most scalable intervention.
This suggests that for most of Namibia water access is not a significant issue, since the best improvement is to increase the filtering of the water.
This result matched the expectation given by previous research that the universal change that would most improve clean water access is chlorine filtering******.

In $x$ of the population clusters, a hand dug well was the most cost-effective best intervention.
These $x$ clusters [change to an exact amount] returned that a hand dug well would be the best intervention.
And compared to the rest of country, the clusters in this group were the overall most cost-effective locations for an intervention. 
[Include in discussion how this supports the idea of the effectiveness of the model, also the following sentence could be moved to the discussion and expanded upon]
This suggests that these clusters lack close, improved water sources, and a large amount of good would come from focusing improvement efforts there.

[Does this belong in the discussion section?]
As this method centers around calculating the Disability Adjusted Life Year, a quantity that is difficult to truly measure, it is calculated using numeric assumptions that have been gathered from research, or estimated by survey data (see Appendix A).
While the accuracy of each of these assumptions is based on the accuracy and amount of available research, it is possible to test the sensitivity of the results on these values.

The sensitivity of a specific variable identified is determined by whether or not it could significantly change the outcome of the model.
That sensitivity is calculated by raising the variable by 10 percent, and then lowering it by 10 percent, and observing the impact of this change on the overall results.
A strong sensitivity to a certain variable (determined by a significant change in the interventions average ranking) would suggest that great care and further research must occur in finding its true value.
Low sensitivity, however, suggests the model is resilient to errors caused by estimating the value of the variable, and the ranking will likely be the same whether the true value was used, or the approximation provided.
For example, an important variable for determining the benefit of an intervention is the fraction of individuals that will seek medical attention after experiencing diarrhea.
Based on literature, this value should be be approximately 25 percent (see Appendix A).
To test the sensitivity of our model to this assumption, we ran the model twice with adjusted values for this variable, as described.
The outcome of this analysis was that the change in the ranking of a given cluster was, on average, less than one.
Thus, even if the percentage of people who visit a doctor is truly 27.5 percent, the cluster rankings will still be practically identical to the results achieved at a value of 25 percent.
For each of the variables tested here are the level of sensitivity between each variable and the cost and benefits.

\begin{center}
\begin{tabular}{l|l|l}
\hline
& Costs & DALYs Saved \\ \hline
Portion of cases that receive treatment & X & X \\ \hline
Rural improvement from new treatment & X & XX \\ \hline
Urban improvement from new treament & XXX & XXX \\ \hline
Relative improvememnt from dug well & X & X \\ \hline
Relative improvement from tube well & X & X \\ \hline
Relative improvement from Standpost & X & X \\ \hline
Individuals per household & X & XX \\ \hline
Urban chlorine recurring costs & XXX & XX \\ \hline
Rural chlorine recurring costs & XX & X \\ \hline
Waters trips per day & X & X \\ \hline

\end{tabular}
\end{center}

First topic of discussion, why were clusters that needed dug wells the most important places for an intervention. 

[Combine following lines with previous since we are combining results and discussion]
Dug wells were significantly more expensive than chlorine filters, requiring large capital costs and non trivial cost for upkeep and repairs. 
Therefore, it is to be expected that dug wells would only be recommended for clusters where there is limited access to any kind of water. 
The model followed this pattern very well.
Clearly, chlorine filters do little to solve this problem of limited access, so the water needs of the people will only truly be met by creating a new source of water. 
These circumstances would certainly explain why the clusters that showed a dug well as their most cost-effective intervention would also be the most cost-effective locations to work in. These clusters are simply the most in need of an intervention.  

It is very interesting to note how very selective the model was in choosing the dug well intervention versus the chlorine filters.
Out of 550 cluster, the model suggested a dug well as the most cost-effective intervention only $x$ times, or $p$ percent of the time.
This result is a measure of the overall access to improved water for Namibia. 
Namibia already has a fairly good access to improved water, so this result build confidence in the validity of the models results.