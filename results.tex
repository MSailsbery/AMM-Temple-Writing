We focused the testing of our model on the country of Namibia as case study.
While Namibia is a sub-saharan African country, it already has fairly good water resources available to the majority of its citizens.
So this made the country a great place to start since our goal to simply find the regions and communities that are still in need of a water intervention.  

The first aspect of the results that stood out was that in almost every cluster the recommended water intevention was chlorine filters.
This was the cheapest intervetion, and this suggests that for most of Namibia, water access is not a significant issue.
Hence, improved water quality through filtering is the most cost effective improvement to make in these areas.
This result matched with our expectations and with previous research which shows that the universal change that would most improve clean water access is chlorine filtering.
But as our research focused on finding the best intervention for a given cluster, we also had a few clusters return a result other than water.
These handful of clusters (how many) returned the dug well as their best water intervention.
More insterestingly, these clusters that showed the dug well as their best intervention were also the most highly ranked in terms of benefit to cost ratio.  

Since our mathematical model is focused on calculating a quantity that is difficult to truly measure, the Disability Adjusted Life Year, it is necessarily built using several numeric assumptions on more basic quantities that are easier to estimate.
We cannot easily test the validity of each of these assumptions, but we can test our model's sensitivity to the choice for each value.
With each sensitivity test, we adjust a specific variable that might sway change the outcome of the model.
We test whether raising it by 10 percent or lowering it by 10 percent has a significant impact on the overall results.
A strong sensitivty to any certain variable would suggest that great care and further research must occur in finding its value.
And low sensitivity suggests the model is resilient to errors in estimating the value of the variable.
For example, an important variable for determining the benefit of an intervention is the fraction of individuals that will seek medical attention after experiencing diarrhea.
Based on literature, we estimated this value at 25 percent.
To test the sensitivity of our model to this assumption, we ran the model twice with adjusted values for this variable; once with it lowered 10 percent, the other raised 10 percent.
In the results we saw...  

