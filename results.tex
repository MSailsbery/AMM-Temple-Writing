We focused the testing of our model on the country of Namibia as case study.
While Namibia is a sub-saharan African country, it already has fairly good water resources available to much of its citizens.
So this made the country a great place to start since our goal to simply find the regions and communities that are still in need of a water intervention.  

The first aspect of the results that stood out was that in almost every cluster the recommended water intevention was chlorine filters.
This was the cheapest intervetion, and this suggests that for most of Namibia, water access is not a significant issue.
Hence, improved water quality through filtering is the most cost effective improvement to make in these areas.
This result matched with our expectations and with previous research which shows that the universal change that would most improve clean water access is chlorine filtering.
But as our research focused on finding the best intervention for a given cluster, we also had a few clusters return a result other than water.
These handful of clusters (how many) returned the dug well as their best water intervention.
More insterestingly, these clusters that showed the dug well as their best intervention were also the most highly ranked in terms of benefit to cost ratio.
