The method used to predict the benefit gained from a water infrastructure intervention [define this in intro] is an example of a cost-benefit analysis. Using the framework developed in Valuing Water, Valuing Livelihoods*, we optimize the amount of time gained by citizens per the amount of money spent.

[Include a 20-liner on the history of cost benefit analysis in civil engineering or more specifically hydrology]

For any given water intervention, the exact implementation must be decided after a thorough visit to the community it is intended to serve.
This includes an on-site expert who will speak with local leaders and community members, observe and evaluate the current water system, and determine where and how the decided intervention should be implemented.
Does the community even desire the intervention suggested?
Will they be able to maintain it well, or will it need to be replaced after only a few years?
Will the intervention actually improve the current water situation, either by reducing time needed to get water, or improving water quality?
These questions, among others, are all things that can only be determined by an on-site expert.
As such, the model discussed is not a substitute for current methods of evaluating the best way to help in-need communities, but rather a way to decide what communities should be priorities for visits and evaluations.
[Include a 15-liner that talks about the research of the community that is frequently mentioned in Valuing Water, Valuing Livelihoods]

This methodology is incapsulated in the article by Uneze et. al on the Bauchi State of Nigeria. (Eberechukwu Uneze , Ibrahim Tajudeen \& Ola Iweala (2012) Cost- effectiveness and benefit?cost analyses of some water interventions in Nigeria: the case of Bauchi State, Journal of Development Effectiveness, 4:4, 497-514, DOI: 10.1080/19439342.2012.716075)
In order to do an effective cost-benefit and cost-effectiveness analysis on the state, local departments of health and water were contacted, region specific past data was collected, and after a great deal of compilation, an effective method was decided.
Its methodology included data specific to the local organizations of government, experimental values of the surrounding countries, and a sensitivity analysis on the robustness of the assumptions required.
This facilitated a very accurate analysis for that specific state.
Although this particular study is a stupendous example of the power of the WHO cost-benefit analysis, it requires a great deal of time and effort to be accomplished.
In order to give additional guidance on a broader scale and to be able to use accurate generalizations on a national or regional level, a more expansive method is needed.