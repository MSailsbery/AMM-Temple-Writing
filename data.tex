With the advent of improved data availability, generalized data-driven modeling of water availability has become a simpler task. The three major statistics that were used to calculate the efficacy of each intervention were the travel time for each individual to their nearest water source, the quality of water that each individual receives, and the intervention cost. The data to estimate these statistics was, by majority, from a survey conducted by the DHS program, with other national statistics and data from water availability literature included as well. 
The DHS program survey was a cluster sampling survey, so, while individual responses from survey participants were recorded, some data such as the cluster location was only recorded on a cluster level. Thus, the intervention efficacy was only predicted on a cluster by cluster level. As the clusters were much more numerous than the regional or subregional political divisions, this restriction is not significant.
