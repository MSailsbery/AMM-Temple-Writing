With the advent of improved data availability, generalized data-driven modeling of water availability has become a simpler task.* The three major statistics that were used to calculate the efficacy of each intervention were the travel time for each individual to their nearest water source, the quality of water that each individual receives, and the intervention cost. The data to estimate these statistics was, by majority, from a survey conducted by the DHS program, with other national statistics and data from water availability literature included as well.*
The DHS program survey was a cluster sampling survey, so, while individual responses from survey participants were recorded, some data such as the cluster location was only recorded on a cluster level. Thus, the intervention efficacy was only predicted on a cluster by cluster level. As the clusters were much more numerous than the regional or subregional political divisions, this restriction is not significant.*
The travel time, $t_{c,i}$, of each of the surveyed individuals from each cluster, $c$, was recorded as a survey question. In this model, these times are assumed to be representative of the population of each cluster as a whole. However, the individuals surveyed were not picked randomly, so this assumption introduces some inaccuracies into the model [develop this idea more. bring closure to it].
The quality of water, $q_{c,i}$, of each of the surveyed individuals from each cluster, $c$, was estimated using survey questions about prevalence of diarrhea in the participants household, as well as their type of water source and water purifying habits. The effect of water purification on the relative risk of disease was estimate using Valuing Water, Valuing Livelihoods*. As with the travel time, the population that was surveyed in the cluster was assumed to be representative of the entire population of the cluster.
The cost of interventions was taken from a report on costs of water infrastructure improvements by the World Health Organization*. 