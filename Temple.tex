%Formatting taken in part from Journal article template by Frits Wenneker at www.LaTeXTemplates.com


\documentclass[twoside,twocolumn]{article}

\usepackage{blindtext} % Package to generate dummy text throughout this template 
\usepackage{amsmath}

\usepackage[T1]{fontenc} % Use 8-bit encoding that has 256 glyphs

\usepackage{microtype} % Slightly tweak font spacing for aesthetics

\usepackage[english]{babel} % Language hyphenation and typographical rules

\usepackage[hmarginratio=1:1,top=32mm,columnsep=20pt]{geometry} % Document margins
\usepackage[hang, small,labelfont=bf,up,textfont=it,up]{caption} % Custom captions under/above floats in tables or figures
\usepackage{booktabs} % Horizontal rules in tables

\usepackage{lettrine} % The lettrine is the first enlarged letter at the beginning of the text

\usepackage{enumitem} % Customized lists
\setlist[itemize]{noitemsep} % Make itemize lists more compact

\usepackage{abstract} % Allows abstract customization
\renewcommand{\abstractnamefont}{\normalfont\bfseries} % Set the "Abstract" text to bold
\renewcommand{\abstracttextfont}{\normalfont\small\itshape} % Set the abstract itself to small italic text

\usepackage{titlesec} % Allows customization of titles
\renewcommand\thesection{\Roman{section}} % Roman numerals for the sections
\renewcommand\thesubsection{\roman{subsection}} % roman numerals for subsections
\titleformat{\section}[block]{\large\scshape\centering}{\thesection.}{1em}{} % Change the look of the section titles
\titleformat{\subsection}[block]{\large}{\thesubsection.}{1em}{} % Change the look of the section titles

\usepackage{fancyhdr} % Headers and footers
\pagestyle{fancy} % All pages have headers and footers
\fancyhead{} % Blank out the default header
\fancyfoot{} % Blank out the default footer
\fancyhead[C]{LDS TEMPLE LOCATION AND MEMBER ACCESSIBILITY $\bullet$ May 2017} % Custom header text
\fancyfoot[RO,LE]{\thepage} % Custom footer text

\usepackage{titling} % Customizing the title section

\usepackage{hyperref} % For hyperlinks in the PDF

%----------------------------------------------------------------------------------------
%	TITLE SECTION
%----------------------------------------------------------------------------------------

\setlength{\droptitle}{-4\baselineskip} % Move the title up

\pretitle{\begin{center}\Huge\bfseries} % Article title formatting
\posttitle{\end{center}} % Article title closing formatting
\title{LDS Temple Location and Member Accessibility} % Article title
\author{%
\textsc{Applied Mathematical Modeling Research Team}\\[1ex] % Your name
\normalsize Brigham Young University  \\ % Your institution
\normalsize \href{mailto:amm@math.byu.edu}{amm@math.byu.edu} % Your email address
%\and % Uncomment if 2 authors are required, duplicate these 4 lines if more
%\textsc{Jane Smith}\thanks{Corresponding author} \\[1ex] % Second author's name
%\normalsize University of Utah \\ % Second author's institution
%\normalsize \href{mailto:jane@smith.com}{jane@smith.com} % Second author's email address
}
\date{\today} % Leave empty to omit a date
\renewcommand{\maketitlehookd}{%
\begin{abstract}
\noindent This paper uses various mathematical models to determine the best locations for the LDS church to build more temples.
These models take into account current temple locations and the density of members of the LDS church in different locations. 
The Euclidean Distance Model is used followed by the Markov Chain Model.
\end{abstract}
}
% Do we need a keywords section?
\begin{document}
\maketitle


%----------------------------------------------------------------------------------------
%	ARTICLE CONTENTS
%----------------------------------------------------------------------------------------

\section{Introduction}

For the Church of Jesus Christ of Latter-Day Saints (LDS church), temples are believed to be imperative in the process of one gaining salvation.
It is consequently important to the LDS church that all members have access to a temple.
Not only do members of the LDS church believe that temples are vital to salvation, but they also view temples as a place of worship (Monson, 1995).
So it is additionally important to be able to provide regular temple access to the maximum number of LDS church members.

The LDS church currently consists of 15 million members ("Facts and Statistics",2017).
In an effort to provide these members with temple access, the LDS church has built a large number of temples and continues to build them.
In fact, the LDS church at this time has 155 temples in operation worldwide. 
They additionally have 14 temples under construction and 13 more that have been announced to be built ("Statistics", 2017).

When building new temples, LDS church leaders have to consider the most effective locations to build them at.
Among many other considerations in this process, is the factor of giving the maximum number of people access or more regular access to a temple.
In the field of operations research, this factor is often considered a facility location problem.

Facility location problems are a highly-research subfield of operations research. 
They generally involve a set of demands in different locations and facilities that can fulfill the demands (Melo, Nickel, \& Saldanha-da-Gama, 2009).
The problem is to choose the optimal locations of these facilities so as to maximize the number of demands filled.

There are a myriad of ways to define what an optimal location would be.
The most common definition is a subset of median problems that measure the average distance traveled to fulfill a demand (Owen \& Daskin, 1998). 
The most prominent and simplest of these types of problems, is the P-median problem.
This problem seeks to minimize the total distance that a demand must travel to reach its nearest facility (Current, Min, \& Schilling, 1990).
Most other median problems are slight variants of the P-median problem.

In our paper, we use similar mathematical models to solve the temple location problem of the LDS church. 
We use the Euclidean Distance Model to find the optimal temple locations in regards to the problem of maximizing the number of church members that have access to a temple. 
Additionally, we use a Markov chain model to find the optimal temple locations in regards to the problem of maximizing the number of church members that have more frequent access to the temple. 
Lastly, we will use a ... % TODO- write about new model

The rest of the paper will be as follows.
First, Section \ref{sec:litreview} contains a literature review on relevant literature.
Section \ref{sec:prob} will present the problem in terms of a facility location problem.
It will define variables and discuss the assumptions made about the data gathered.
Section \ref{sec:models} will examine three modeling approaches to the problem.
Namely, a Euclidean distance model, a Markov chain model, and ... %TODO- Decide on last model.
Section \ref{sec:res} will present the results and Section \ref{sec:analysis} will analyze the implications of the results.
Finally, Section \ref{sec:conclusion} will give a summary of the paper along with areas for further research.

\section{Relevant Literature Review}
\label{sec:litrev}
Facility location management is a well-established area within operations research. Current, Min and Schilling report that facility location management has been researched through the perspective of many different academic disciplines. They additionally cite one recent bibliography on the subject had over 1500 titles, showing just how expansive research in the area is (Current, Min, \& Schilling). In their literature review, Owen and Daskin give an overview of the most used models among this immense amount of literature. They comment on 15 differing subsets of facility location problems, again demonstrating the voluminousness of the literature (Owen \& Daskin). 

Although the library of facility location management is vast, this paper focuses primarily on literature pertaining to median problems, with a specific emphasis on using Markov chain models. In the past, no model has been presented to optimize the placement of LDS temples, in either the United States or internationally. Our paper seeks to fill this gap in the literature.

The study of location theory began with Alfred Weber in 1909. 
He researched where to place a warehouse based off of the average distance that would be traveled by those who would need to visit the warehouse (Owen \& Daskin).
This research created the basis for {\em median problems}, or problems to determine a facility location by minimizing the average distance traveled by consumers.
As noted by Melo, Nickel and Saldanha-da-Gamma, the p-median problem is considered the simplest and most prominent median problem. They define the p-median problem as the problem where p facilities need to be placed to minimize the total weighted distance or costs to satisfy customer demands (Melo, Nickel \& Saldanha-da-Gamma).

From the simplest definition of p-median problems arises a myriad of slight variations of it.
These variations are very commonly used to solve median problems.
For example, Berman, Drezner, and Wesolowsky use a model formulation similar to the p-median problem except that demands can be serviced by facilities other than the one closest to them (Berman, Drezner \& Wesolowsky, 2003).

\section{Problem Formulation}
\label{sec:prob}
Insert text here.
\subsection{Important Variables}
\begin{tabular}{c | l}
Variable & Description\\
\hline
$d(s_{i},t_{j})$ & Distance from stake $s_{i}$ to temple $t_{j}$\\
$\rho_{i}$ & Density of stake $s_{i}$\\
$S$ & Set of all stakes\\
$s_{i}$ & Stake number $i$\\
$T$ & Set of all Temples\\
$T_{new}$ & The proposed new temple location\\
$t_{i}$ & Temple number $i$ \\
$\tau_{i}$ & Temple score of temple $t_{i}$\\
\end{tabular}
To fully understand our approach, it is crucial to know what variables were considered relevant, and which were excluded.
Obviously, chief among these variables is the euclidean distance $d(s_{i},t_{j})$ from a stake center to a temple.
While this may be sufficient for the euclidean distance model, a fair number of other considerations were necessary for the Markov chain model and the ... .
For the Markov chain model, we included a measure of the busyness for each temple.
To do this, we gave each temple a score, $\tau_{i}$, based on the number of stakes that listed that temple as one of it's five closest temples.
Using these scores, we assigned each stake a density $\rho_{i}$, that sought to relate how much traffic each stake would provide to it's surrounding temples.
This allowed us to approximate how busy a temple was, using the stakes that consider it one of their closest temples.
\subsection{Assumptions and Data}
To simplify the models and make an analytical review of temple location optimization possible, a number of assumptions were made. Firstly, it is assumed that every LDS stake in the United States has the same number of active, temple-going members contained within its boundaries. This approximation, combined with the assumption that every stake center is located near the center of the stake, allows us to estimate the member density throughout the country. Within each model, further assumptions are made regarding the proportion of members that attend daily from each stake, and how the distance and busyness of each temple affects these proportions, but these details will be clarified for each individual model.

\section{Modeling Approaches}
\label{sec:models}
This section is a mathematical explanation of the assumptions made and the models used. As choosing the location of an LDS temple mathematically involves various assumptions about human behavior, as well as several possible objectives, we separate our explanation of the models used to solve this problem into subsections. Each subsection is self-contained in the assumptions made and the problem setup. Along with the individual models presented in each subsection, there is also a brief summary of the algorithmic approach used to find the optimal solution to each.

\subsection{Euclidean Distance Model} %TODO - Name needs to be changed as we are not in Euclidean space

The simplest model used is the Euclidean Distance Model (EDM). The purpose of EDM is to minimize the average distance a given member of a LDS stake has to travel to get to their closest temple. As such, this model does not take into account the capacity of the temples, nor the relative accessibility of other temples, even if they are a similar distance away. The only assumption made about the behavior of church members are that they desire to have a temple close to their stake center, and that they will only use the closest temple to them.

The mathematical setup is as follows. Let each stake be represented as $s_i$ and each temple $t_j$, and let the set of all the stakes be represented as $S$ and likewise the temples as $T$. The average distance a given member of an LDS stake has to travel to get to their closest temple is, 

\begin{equation}
	f(S,T) = \sum_i \underset{j}{\text{min}}\{d(s_i,t_j)\}.
\end{equation}

The unconstrained optimization problem can then be written as,

\begin{equation}
\begin{aligned}
	\underset{T_{new}}{\text{Minimize }} f(S,T^*), \text{ where } T^* = T \cup T_{new}.
\end{aligned}
\end{equation}

Insert more text here.


\subsection{Markov Modeling}

Markov chains are used in a myriad of fields. For the purposes of this study, Markov chains are useful because they reveal the effects that simple local behaviors have on more complex global behaviors. Specifically, the effects of the busyness of each individual temple on the members that live nearby it can be modeled on a more global level.

This section gives a brief outline of key characteristics of a Markov chain and then the specific application of those in this model.

\subsubsection{Background}

A Markov chain is an example of...

\subsubsection{Model}
The second model used is closely based on the Markov chain application of (reference paper). It involves two stages: the initialization of the model transition matrix and the computation of the Perron eigenvectors, Mean First Passage Time, and Kemeny constant described above. To understand this concept more fully, we consider a graph in which each stake is connected to the five closest temples as shown in \ref{}.
% Create a figure to demonstrate stakes connected to 5 closest temples

In order to begin creating this model, we initialize a square probability matrix $T$ of zeros in which the first $n$ rows and columns correspond to the church stakes in the United States and the next $m$ rows and columns correspond to the church temples in the United States. Each $t_{ij}$ in the matrix represents the probability that a person in row $i$ will travel to column $j$ in the current state. The diagonal entries $t_{ii}$ represent the probability that a person will remain in their current location through the current state. The states of the matrix represent a single day.
% Create a visual matrix representation of the matrix initialized

We begin populating the matrix by allowing the diagonals $t_{ii} = 29/30$ which numerically represents the assumption that in each state or day, $1/30$ of a stake's active temple attendees will attend the temple. We then adjust this number with a measure of a stakes relative busyness in a single day. To do so, we compute a measure of busyness by taking into account the number of stakes within a close proximity to each temple. The required numerical representations are described below.

\begin{equation}
\begin{aligned}
Ts_j = \sqrt{1\over{\sum_j{t_ij}}}
\end{aligned}
\end{equation}


Insert text here.
\subsection{Last Model} % TODO- figure out last model. Once updated the paper overview in the introduction will need to be updated.
Insert text here.

\section{Results}
Insert text here.
\label{sec:res}
Insert text here.
\subsection{Euclidean Distance Model}
Insert text here.
\subsection{Markov Model}
Insert text here.
\subsection{Last Model}
Insert text here.

\section{Analysis}
\label{sec:analysis}
Insert text here.

\section{Conclusion}
\label{sec:conclusion}
Insert text here.

%----------------------------------------------------------------------------------------
%	REFERENCE LIST
%----------------------------------------------------------------------------------------

\begin{thebibliography}{99} % Bibliography - this is intentionally simple in this template
% Below is an example
%\bibitem[Crisostomi, Kirkland and Shorten, 2011]{Crisostomi:2011dg}
%Crisostomi E, Kirkland S, Shorten R. 
%\newblock A Google-like model of road network dynamics and its application to regulation and control.
%\newblock {\em Int. J. Control.} 2011;84:633--651.

\bibitem[Berman, Drezner, Wesolowsky, 2003]{Berman:2003}
Berman O, Drezner Z, Wesolowsky G.
\newblock Locating service facilities whose reliability is distance dependent.
\newblock {\em Computers \& Operations Research.} 2003; 30:1683-1695.

\bibitem[Current, Min, Schilling, 1990]{Current:1990}
Current J, Min H, Schilling D.
\newblock Multiobjective analysis of facility location decisions.
\newblock {\em European Journal of Operational Research.} 1990; 49:295-307.

\bibitem["Facts and Statistics", 2017]{Facts&Stats:2017}
\newblock 2017; Retrieved from http://www.mormonnewsroom.org/facts-and-statistics

\bibitem{Melo, Nickel, & Saldanha-da-Gama, 2009}
Melo M, Nickel S, Saldanha-da-Gama F.
\newblock Facility location and supply-chain management - A review.
\newblock {\em European Journal of Operational Research.}  2009; 196:401--412.

\bibitem[Monson, 1995]{Monson:1995}
Monson T.
\newblock Blessings of the Temple. 1995;
\newblock Retrieved from https://www.lds.org/church/temples/why-we-build-temples/blessings-of-the-temple?lang=eng

\bibitem[Owen, Daskin, 1998]{Owen:1998}
Owen S, Daskin M.
\newblock Strategic facility location: a review.
\newblock {\em European Journal of Operational Research.} 1998; 111:423-447.

\bibitem["Statistics", 2017]{Stats:2017}
\newblock 2017; Retrieved from http://www.ldschurchtemples.com/statistics/
\end{thebibliography}

%----------------------------------------------------------------------------------------

\end{document}
