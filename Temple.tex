%Formatting taken in part from Journal article template by Frits Wenneker at www.LaTeXTemplates.com


\documentclass[twoside,twocolumn]{article}

\usepackage{blindtext} % Package to generate dummy text throughout this template 
\usepackage{amsmath}

\usepackage[T1]{fontenc} % Use 8-bit encoding that has 256 glyphs

\usepackage{microtype} % Slightly tweak font spacing for aesthetics

\usepackage[english]{babel} % Language hyphenation and typographical rules

\usepackage[hmarginratio=1:1,top=32mm,columnsep=20pt]{geometry} % Document margins
\usepackage[hang, small,labelfont=bf,up,textfont=it,up]{caption} % Custom captions under/above floats in tables or figures
\usepackage{booktabs} % Horizontal rules in tables

\usepackage{lettrine} % The lettrine is the first enlarged letter at the beginning of the text

\usepackage{enumitem} % Customized lists
\setlist[itemize]{noitemsep} % Make itemize lists more compact

\usepackage{abstract} % Allows abstract customization
\renewcommand{\abstractnamefont}{\normalfont\bfseries} % Set the "Abstract" text to bold
\renewcommand{\abstracttextfont}{\normalfont\small\itshape} % Set the abstract itself to small italic text

\usepackage{titlesec} % Allows customization of titles
\renewcommand\thesection{\Roman{section}} % Roman numerals for the sections
\renewcommand\thesubsection{\roman{subsection}} % roman numerals for subsections
\titleformat{\section}[block]{\large\scshape\centering}{\thesection.}{1em}{} % Change the look of the section titles
\titleformat{\subsection}[block]{\large}{\thesubsection.}{1em}{} % Change the look of the section titles

\usepackage{fancyhdr} % Headers and footers
\pagestyle{fancy} % All pages have headers and footers
\fancyhead{} % Blank out the default header
\fancyfoot{} % Blank out the default footer
\fancyhead[C]{Running title $\bullet$ May 2016 $\bullet$ Vol. XXI, No. 1} % Custom header text
\fancyfoot[RO,LE]{\thepage} % Custom footer text

\usepackage{titling} % Customizing the title section

\usepackage{hyperref} % For hyperlinks in the PDF

%----------------------------------------------------------------------------------------
%	TITLE SECTION
%----------------------------------------------------------------------------------------

\setlength{\droptitle}{-4\baselineskip} % Move the title up

\pretitle{\begin{center}\Huge\bfseries} % Article title formatting
\posttitle{\end{center}} % Article title closing formatting
\title{LDS Temple Location and Member Accessibility} % Article title
\author{%
\textsc{Applied Mathematical Modeling Research Team}\\[1ex] % Your name
\normalsize Brigham Young University  \\ % Your institution
\normalsize \href{mailto:amm@math.byu.edu}{amm@math.byu.edu} % Your email address
%\and % Uncomment if 2 authors are required, duplicate these 4 lines if more
%\textsc{Jane Smith}\thanks{Corresponding author} \\[1ex] % Second author's name
%\normalsize University of Utah \\ % Second author's institution
%\normalsize \href{mailto:jane@smith.com}{jane@smith.com} % Second author's email address
}
\date{\today} % Leave empty to omit a date
\renewcommand{\maketitlehookd}{%
\begin{abstract}
\noindent Markov chains are used to model many different dynamical systems such as the movement of traffic, the spreading of disease, and the volatility of financial assets. In this paper, we construct a Markov model of the distribution of water in water poor areas of the world. Using current data about the water availability to individuals in these areas, we present the efficiency of the distribution process as well as examine various changes to that process and their effects on the efficiency. Then, we present what specific changes to these water distribution systems would be the most significant for the populations that they serve.
% TODO- Abstracts should also include results. So when we finish our paper we may want to change up the last part of the abstract to account for our results. Or if there are too many, the last sentence will suffice. -McKell
\end{abstract}
}
\begin{document}
\maketitle


%----------------------------------------------------------------------------------------
%	ARTICLE CONTENTS
%----------------------------------------------------------------------------------------

\section{Introduction}

The rest of the paper will be as follows.
First, Section \ref{sec:lit_rev} will present a review on modern literature pertaining to facility location problems.
Section \ref{sec:prob} will present the problem in terms of a facility location problem.
Section \ref{sec:sols} will examine three solutions approaches to the problem.
Namely, a euclidean distance model, a Markov chain model, and ... %TODO- Decide on last model or delete this part.
Section \ref{sec:res} will present the results and Section \ref{sec:analysis} will analyze the implications of the results.
Finally, Section \ref{sec:conclusion} will give a summary of the paper along with areas for further research.
% TODO- This section should provide a general overview on any and all of the following: water distribution, facility location research, Markov models, etc. This section will be written in a technical report like fashion. This can function as our literature review and should be chalked full of references. This should be followed by an explicit definition of our problem &/or thesis statement where applicable (The problem doesn't need to be mathematically defined- just the word version). It should end with a paragraph that is like a table of contents for the rest of the paper. - McKell

\section{Relevant Literature Review}
\label{sec:lit_rev}
% TODO- This section is optional. It can all be covered in the beginning of the introduction. But if there is too much literature to cover, we definitely can have this section. Maybe we could just do this section on operational research topics and have our introduction be about water management?-McKell
Insert text here.

\section{Problem Formulation}
\label{sec:prob}
% TODO - This section should explicitly define our problem and variables used. I think it will probably be the P-median problem or a slight modification of it. - McKell
Insert text here.

<<<<<<< HEAD
\section{Modeling Approaches}
\label{sec:models}
This section is a mathematical explanation of the assumptions made and the models used. As choosing the location of an LDS temple mathematically involves various assumptions about human behavior, as well as several possible objectives, we separate our explanation of the models used to solve this problem into subsections. Each subsection is self-contained in the assumptions made and the problem setup. Along with the individual models presented in each subsection, there is also a brief summary of the algorithmic approach used to find the optimal solution to each.

\subsection{Euclidean Distance Model} %TODO - Name needs to be changed as we are not in Euclidean space

The simplest model used is the Euclidean Distance Model (EDM). The purpose of EDM is to minimize the average distance a given member of a LDS stake has to travel to get to their closest temple. As such, this model does not take into account the capacity of the temples, nor the relative accessibility of other temples, even if they are a similar distance away. The only assumption made about the behavior of church members are that they desire to have a temple close to their stake center, and that they will only use the closest temple to them.

The mathematical setup is as follows. Let each stake be represented as $s_i$ and each temple $t_j$, and let the set of all the stakes be represented as $S$ and likewise the temples as $T$. The average distance a given member of a LDS stake has to travel to get to their closest temple is, 

\begin{equation}
	f(S,T) = \sum_i \underset{j}{\text{min}}\{d(S_i,T_j)\}.
\end{equation}

The unconstrained optimization problem can then be written as,

\begin{equation}
\begin{aligned}
	\underset{T_{new}}{\text{Minimize }} f(S,T^*), \text{ where } T^* = T \cup T_{new}.
\end{aligned}
\end{equation}

Insert more text here.


\subsection{Markov Modeling}
Insert text here.
\subsubsection{Background}
Insert text here.
\subsubsection{Model}
Insert text here.
\subsection{Last Model} % TODO- figure out last model or delete this section. Once updated the paper overview in the introduction will need to be updated. - McKell
Insert text here.

\section{Results}
\label{sec:res}
% TODO- This section should be purely the computational results - save the analysis for the next section. - McKell
Insert text here.
\subsection{Euclidean Distance Model}
Insert text here.
\subsection{Markov Model}
Insert text here.
\subsection{Last Model}
Insert text here.

\section{Analysis}
\label{sec:analysis}
% TODO - This section should contain what are results mean and their implications. -McKell
Insert text here.

\section{Conclusion}
\label{sec:conclusion}
% TODO - This section should be about 1/2-1 page long. It should state no new information, but basically a summary of the paper. The "tell them what you told them" part of the paper. It should then include a paragraph about possibilities for further research.- McKell
Insert text here.

%----------------------------------------------------------------------------------------
%	REFERENCE LIST
%----------------------------------------------------------------------------------------

\begin{thebibliography}{99} % Bibliography - this is intentionally simple in this template

\bibitem[Crisostomi, Kirkland and Shorten, 2011]{Crisostomi:2011dg}
Crisostomi E, Kirkland S, Shorten R. 
\newblock A Google-like model of road network dynamics and its application to regulation and control.
\newblock {\em Int. J. Control.} 2011;84:633--651.
 
\end{thebibliography}

%----------------------------------------------------------------------------------------

\end{document}
