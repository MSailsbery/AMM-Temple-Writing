%Formatting taken in part from Journal article template by Frits Wenneker at www.LaTeXTemplates.com


\documentclass[twoside,twocolumn]{article}

\usepackage{blindtext} % Package to generate dummy text throughout this template 
\usepackage{amsmath}

\usepackage[T1]{fontenc} % Use 8-bit encoding that has 256 glyphs

\usepackage{microtype} % Slightly tweak font spacing for aesthetics

\usepackage[english]{babel} % Language hyphenation and typographical rules

\usepackage[hmarginratio=1:1,top=32mm,columnsep=20pt]{geometry} % Document margins
\usepackage[hang, small,labelfont=bf,up,textfont=it,up]{caption} % Custom captions under/above floats in tables or figures
\usepackage{booktabs} % Horizontal rules in tables

\usepackage{lettrine} % The lettrine is the first enlarged letter at the beginning of the text

\usepackage{enumitem} % Customized lists
\setlist[itemize]{noitemsep} % Make itemize lists more compact

\usepackage{abstract} % Allows abstract customization
\renewcommand{\abstractnamefont}{\normalfont\bfseries} % Set the "Abstract" text to bold
\renewcommand{\abstracttextfont}{\normalfont\small\itshape} % Set the abstract itself to small italic text

\usepackage{titlesec} % Allows customization of titles
\renewcommand\thesection{\Roman{section}} % Roman numerals for the sections
\renewcommand\thesubsection{\roman{subsection}} % roman numerals for subsections
\titleformat{\section}[block]{\large\scshape\centering}{\thesection.}{1em}{} % Change the look of the section titles
\titleformat{\subsection}[block]{\large}{\thesubsection.}{1em}{} % Change the look of the section titles

\usepackage{fancyhdr} % Headers and footers
\pagestyle{fancy} % All pages have headers and footers
\fancyhead{} % Blank out the default header
\fancyfoot{} % Blank out the default footer
\fancyhead[C]{Running title $\bullet$ May 2016 $\bullet$ Vol. XXI, No. 1} % Custom header text
\fancyfoot[RO,LE]{\thepage} % Custom footer text

\usepackage{titling} % Customizing the title section

\usepackage{hyperref} % For hyperlinks in the PDF

%----------------------------------------------------------------------------------------
%	TITLE SECTION
%----------------------------------------------------------------------------------------

\setlength{\droptitle}{-4\baselineskip} % Move the title up

\pretitle{\begin{center}\Huge\bfseries} % Article title formatting
\posttitle{\end{center}} % Article title closing formatting
\title{LDS Temple Location and Member Accessibility} % Article title
\author{%
\textsc{Applied Mathematical Modeling Research Team}\\[1ex] % Your name
\normalsize Brigham Young University  \\ % Your institution
\normalsize \href{mailto:amm@math.byu.edu}{amm@math.byu.edu} % Your email address
%\and % Uncomment if 2 authors are required, duplicate these 4 lines if more
%\textsc{Jane Smith}\thanks{Corresponding author} \\[1ex] % Second author's name
%\normalsize University of Utah \\ % Second author's institution
%\normalsize \href{mailto:jane@smith.com}{jane@smith.com} % Second author's email address
}
\date{\today} % Leave empty to omit a date
\renewcommand{\maketitlehookd}{%
\begin{abstract}
\noindent % TODO- write abstract.
\end{abstract}
}
% TODO - do we need a keywords section?
\begin{document}
\maketitle


%----------------------------------------------------------------------------------------
%	ARTICLE CONTENTS
%----------------------------------------------------------------------------------------

\section{Introduction}

For the Church of Jesus Christ of Latter-Day Saints (LDS church), temples are believed to be imperative in the process of one gaining salvation.
It is consequently important to the church that all members have access to a temple.
Not only do members of the church believe that temples are vital to salvation, but they also view temples as a place of worship.
% TODO - add in reference. https://www.lds.org/church/temples/why-we-build-temples/blessings-of-the-temple?lang=eng
So it is additionally important to be able to provide regular temple access to the maximum number of church members.

The LDS church currently consists of 15 million members.
% TODO - add in reference. http://www.mormonnewsroom.org/facts-and-statistics
In an effort to provide these members with temple access, the church has built a large number of temples and continues to build them.
In fact, the LDS church at this time has 155 temples in operation worldwide. 
They additionally have 14 temples under construction and 13 more that have been announced that they will be built.
% TODO -add in reference. http://www.ldschurchtemples.com/statistics/

When building new temples, church leaders have to consider the most effective locations to build them at.
Among many other considerations in this process, is the factor of giving the maximum number of people access or more regular access to a temple.
In the field of operations research, this factor is often considered a facility location problem.

% TODO - add in paragraphs about facility location management. Basically a mini-literature review.

In our paper, we use mathematical models to solve the temple location problem of the LDS church. We use an euclidean distance model to find the optimal temple locations in regards to the problem of maximizing the number of church members that have access to a temple. Additionally, we use a Markov chain model to find the optimal temple locations in regards to the problem of maximizing the number of church members that have more frequent access to the temple. Lastly, we will use a ... % TODO- write about new model if it exists.

The rest of the paper will be as follows.
First, Section \ref{sec:prob} will present the problem in terms of a facility location problem.
Section \ref{sec:sols} will examine three solutions approaches to the problem.
Namely, a euclidean distance model, a Markov chain model, and ... %TODO- Decide on last model or delete this part.
Section \ref{sec:res} will present the results and Section \ref{sec:analysis} will analyze the implications of the results.
Finally, Section \ref{sec:conclusion} will give a summary of the paper along with areas for further research.

\section{Problem Formulation}
\label{sec:prob}
\subsection{Relevant Literature Review}
\subsection{Important Variables}
To fully understand our approach, it is crucial to know what variables were considered relevant, and which were excluded.
Obviously, chief among these variables is the euclidean distance from a stake center to a temple.
While this may be sufficient for the euclidean distance model, a fair number of other considerations were necessary for the Markov chain model and the ... .
For the Markov chain model, we included a measure of the business for each temple.
To do this, we gave each temple a score, $\tau_{i}$, based on the number of stakes that listed that temple as one of it's five closest temples.
Using these scores, we assigned each stake a density $\rho_{i}$, that sought to relate how much traffic each stake would provide to it's surrounding temples.
This allowed us to approximate how busy a temple was, using the stakes that consider it one of their closest temples.
\subsection{Assumptions and Data}

Insert text here.

\section{Modeling Approaches}
\label{sec:models}
This section is a mathematical explanation of the assumptions made and the models used. As choosing the location of an LDS temple mathematically involves various assumptions about human behavior, as well as several possible objectives, we separate our explanation of the models used to solve this problem into subsections. Each subsection is self-contained in the assumptions made and the problem setup. Along with the individual models presented in each subsection, there is also a brief summary of the algorithmic approach used to find the optimal solution to each.

\subsection{Euclidean Distance Model} %TODO - Name needs to be changed as we are not in Euclidean space

The simplest model used is the Euclidean Distance Model (EDM). The purpose of EDM is to minimize the average distance a given member of a LDS stake has to travel to get to their closest temple. As such, this model does not take into account the capacity of the temples, nor the relative accessibility of other temples, even if they are a similar distance away. The only assumption made about the behavior of church members are that they desire to have a temple close to their stake center, and that they will only use the closest temple to them.

The mathematical setup is as follows. Let each stake be represented as $s_i$ and each temple $t_j$, and let the set of all the stakes be represented as $S$ and likewise the temples as $T$. The average distance a given member of a LDS stake has to travel to get to their closest temple is, 

\begin{equation}
	f(S,T) = \sum_i \underset{j}{\text{min}}\{d(S_i,T_j)\}.
\end{equation}

The unconstrained optimization problem can then be written as,

\begin{equation}
\begin{aligned}
	\underset{T_{new}}{\text{Minimize }} f(S,T^*), \text{ where } T^* = T \cup T_{new}.
\end{aligned}
\end{equation}

Insert more text here.


\subsection{Markov Modeling}
Insert text here.
\subsubsection{Background}
Insert text here.
\subsubsection{Model}
Insert text here.
\subsection{Last Model} % TODO- figure out last model or delete this section. Once updated the paper overview in the introduction will need to be updated. - McKell
Insert text here.

\section{Results}
\label{sec:res}
% TODO- This section should be purely the computational results - save the analysis for the next section. - McKell
Insert text here.
\subsection{Euclidean Distance Model}
Insert text here.
\subsection{Markov Model}
Insert text here.
\subsection{Last Model}
Insert text here.

\section{Analysis}
\label{sec:analysis}
% TODO - This section should contain what are results mean and their implications. -McKell
Insert text here.

\section{Conclusion}
\label{sec:conclusion}
% TODO - This section should be about 1/2-1 page long. It should state no new information, but basically a summary of the paper. The "tell them what you told them" part of the paper. It should then include a paragraph about possibilities for further research.- McKell
Insert text here.

%----------------------------------------------------------------------------------------
%	REFERENCE LIST
%----------------------------------------------------------------------------------------

\begin{thebibliography}{99} % Bibliography - this is intentionally simple in this template

\bibitem[Crisostomi, Kirkland and Shorten, 2011]{Crisostomi:2011dg}
Crisostomi E, Kirkland S, Shorten R. 
\newblock A Google-like model of road network dynamics and its application to regulation and control.
\newblock {\em Int. J. Control.} 2011;84:633--651.
 
\end{thebibliography}

%----------------------------------------------------------------------------------------

\end{document}
