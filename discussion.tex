First topic of discussion, why were clusters that needed dug wells the most important places for an intervention. 
Dug wells were significantly more expensive than chlorine filters, requiring large capital costs and non trivial cost for upkeep and repairs. 
Therefore, it is to be expected that dug wells would only be recommended for clusters where access to any kind of water is limited. 
And this was the case with our model.
Clearly, chlorine filters do little to solve this problem of limited access, so the water needs of the people will only truly be met by creating a new source of water. 
These circumstances would certainly explain why the clusters that showed a dug well as their most cost-effective intervention would also be the most cost-effective locations to work in. These clusters are simply the most in need of an intervention.  

It is very interesting to note how very selective the model was in choosing the dug well intervention versus the chlorine filters.
Out of 550 cluster, the model suggested a dug well as the most cost-effective intervention only $x$ times, or $p$ percent of the time.
This result is a measure of the overall access to improved water for Namibia. 
Namibia already has a fairly good access to improved water, so this result build confidence in the validity of the models results.